\chapter{Related Work}

There is a wide range of articles discussing DDoS mitigation. Osanaiya et al. \cite{DDoSOverview} have analyzed 96 publications about DDoS. Out of 36 DDoS attack defenses described in the publications, 6 of them are distributed, while 26 of them are installed on the access point. 

DefCOM \cite{DefCOM} is a peer-to-peer framework for collaborative DDoS Defense that is being installed on multiple nodes in one network. The framework has a distributed design with the purpose of splitting up tasks. Each peer in the network can have up to 3 tasks: Classifying, Rate Limiting and Alert Generation - nodes in a network can therefore only perform the tasks that they are good at. The framework does also supports prioritizing of messages. DefCOM is intended for use within an organization and does not propose a solution for inter-organization sharing. It does not contain a new kind of DDoS response mechanism, but builds a lightweight framework for communication between nodes.

A proposal submitted to the Internet Engineering Task Force (IETF) \cite{IETFDraft} describes a peer-to-peer protocol called "DDoS Open Threat Signaling" (DOTS) for signaling source IP addresses of DDoS attacks. The protocol that the authors propose communicates over HTTPS and is REST-API-based, which is the main difference to the solution proposed in this thesis. The communication can be intra- or  inter-organizational. DOTS implements handshake calls, sending mitigation requests with various parameters, and reporting on the efficacy. The draft expired on June 30th 2017 because of inactivity according to IETF guidelines.

This thesis aims to further develop the idea laid out by the paper "A Blockchain-based Architecture for Collaborative DDoS Mitigation with Smart Contracts and SDN" \cite{OriginalPaper} (further called 'Original Paper'). The original paper proposes to use the Ethereum blockchain to signal DDoS attacks and demonstrates a proof of concept smart contract that allows storage of IP addresses. 
