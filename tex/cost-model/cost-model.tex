\section{Cost model}

Define the set of operations that a transaction executes as $ \sigma = (\sigma_1, ..., \sigma_n) $.

The corresponding fees are $ f = (f_1, ..., f_n), \forall f_i \in G $, where $ G $ is the fee schedule defined in the Ethereum Yellow paper. The amount of gas a transaction consumes is shown in Equation \ref{eq:TransactionGasConsumption}:

\begin{equation}
C = \sum_{i = 1}^n \sigma_i \cdot f_i
\label{eq:TransactionGasConsumption}
\end{equation}

The value of $ C $ is best determined by using the \texttt{estimateGas()} function made available by Ethereum clients.

A transaction uses at least 21'000 gas and the maximum amount of gas that can be used in a transaction is 3'141'592 (set to be raised to 5'500'000 in the next fork \cite{EIP150}). At the moment, this means that the condition in Equation \ref{eq:CostCondition} is always fulfilled.

\begin{equation}
    C = [21000, 3141592]
    \label{eq:CostCondition}
\end{equation}

The price is denoted as the variable $ \alpha $. Since according to ethgasstation.info, there are 3 possible speeds, three constants are defined representing the minimum cost to pay in order to get that speed (Equations \ref{eq:GasPriceCheap} - \ref{eq:GasPriceFast}).
\begin{equation}
    \alpha_{cheap} = 2 \cdot 10^{-9} Ether
    \label{eq:GasPriceCheap}
\end{equation}
\begin{equation}
    \alpha_{average} = 20 \cdot 10^{-9} Ether
    \label{eq:GasPriceAverage}
\end{equation}
\begin{equation}
    \alpha_{fast} = 28 \cdot 10^{-9} Ether
    \label{eq:GasPriceFast}
\end{equation}

The price of an ether is noted as $ ETH $. As mentioned previously, the deviations of different compiler are negligible.

Multiplying these values together results in the total cost (Equation \ref{eq:TotalCost}). 
\begin{equation}
 C \cdot \alpha \cdot ETH
 \label{eq:TotalCost}
\end{equation}  For example, a contract that costs 500'000 gas to deploy and is priced at $ \alpha = \alpha_{average} $ per gas, with the price for Ether being \$50, the cost to deploy that contract is \$0.025 (Equation \ref{eq:CostExample}).

\begin{equation}
    500'000 \cdot 10^{-9} \cdot \$50 = \$0.025
    \label{eq:CostExample}
\end{equation}
