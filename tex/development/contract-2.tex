\section{Contract 2: Pointer to web resource}
The main disadvantage of storing reports directly in the contract is that the cost of the data entry scales linearly with the number of reports. What is just a few cents in gas fees for a few reports, can grow expensive at scale.
Ethereum disincentivizes the storage of large data because each node needs to keep track of a whole blockchain by downloading it. 

The second variant of the smart contract works around the space constraints and the big cost of the first variant by storing the list of IP addresses on a web resource and pointing to it on the blockchain. The advantage of this variant is that it works on a much larger scale and is expected to be cheaper in the long-term. The disadvantages are that a web server needs to be set up and a separate specification has to be defined for the format of the web resource. This solution is also prone to connectivity issues and does not take full advantage of the decentralization and immutability of the blockchain.

\subsection{Smart Contract}
The basic data structure of the smart contract is similar to the array store contract. The difference is that the items do not contain IP addresses, but URLs which point to lists of IP addresses. Since it is possible to include as many reports as desired in a web resource, the assumption is made that only 1 pointer is needed per customer. This assumption can simplify the contract, so that no array is needed and that is not necessary anymore to include a helper function that flattens the data.

\lstinputlisting{snippets/ip-pointer-store-outline.sol}

Instead of using an array to store the pointers, a \texttt{mapping(address => Pointer)} is being used. It is the equivalent of a \texttt{Map<address, Pointer>} type in Java, but it uses a 'Javascript object'-like syntax for getting, setting and deleting values.

With this design, each user of the smart contract can have one pointer at a time. With each transaction the previous pointer is overwritten, hence the method name \texttt{setPointer}.

The struct \texttt{Pointer} does not nest the \texttt{IPAddress} struct inside the \texttt{Report} struct like in the previous contract, but instead stores IP address base and mask directly. Although it would be a cleaner design, it would trigger an error message \texttt{Internal type is not allowed for public state variables}. The reason for this that each Ethereum contract has a JSON interface called Application Binary Interface (ABI) in which a structure like this cannot be represented at the moment, therefore this type of structure is not supported.

For this reason, the contract is programmed to do without nested structs, but with 'mappings' instead, another Solidity language feature. The benefit of mappings is that no helper methods are needed to retrieve the data.

Verifying that the sender of the transaction is a customer works by comparing \texttt{members[msg.sender].ip} to 0. In many other programming languages, \texttt{members[msg.sender]} would be compared to \texttt{null} and the comparison the above code would be susceptible to a null-pointer exception. However, in Solidity, there is no concept of 'null'. Instead, accessing a primitive value that has not been set returns zero and accessing a struct that has not been set returns a struct where all values are zero.

\subsection{Web resource}
Several design aspects need to be considered for the web resource. By storing a URL in the blockchain, it is already implied that the connection to the resource is made using HTTP(S), which is a suitable protocol.

A syntax and a data structure that the reports are represented in needs to be selected. It is desired to use an already common syntax like XML, CSV or JSON, since there are a selection of clients for many languages available and because they are heavily standardized.

The use of an established syntax increases the \textbf{Portability} of the protocol. An essay by Nicolas Seriot \cite{ParsingJSON} shows the challenges of covering edge cases in syntax standards by showing inconsistencies in trailing commas, unclosed structures, duplicate keys and white space in JSON, making a case against developing an additional format.

The JSON and XML syntaxes allow to extend a schema by adding more keys to an object (JSON) or by adding more allowed tags to the schema (XML). \textbf{Upgradeability} is a desired property of the protocol, because it allows the web resource format to be developed further in the future to enable more features with backwards compatibility. This is more difficult with a two-dimensional design like CSV, because the format can only be extended by adding more columns.

With the expectation that lists of reports can become very large, it is beneficiary to have the format in a way where it can already be partially evaluated while not yet fully loaded. This is called \textbf{Streaming}.

Streaming in CSV is simple: as soon as a newline character is detected, the client can safely assume that an entry has been fully loaded and that it can be processed. On the contrary, with XML or JSON, who have closing tags, streaming is not possible.

Neither CSV, JSON or XML have both good Streamability and Upgradeability, and other formats don't have good Portability.

Line delimited JSON streaming \cite{LineDelimitedJSON} provides a reasonable tradeoff. Its syntax is a list of items that are delimited using a newline character (\texttt{{\textbackslash}n}), where each item is a valid JSON object according to the JSON standard RFC 7158 \cite{RFC7158}. Packages for line-delimited JSON streaming exist for at least node.js and Python in their respective registries, so client libraries are available. As a fallback, it is always possible to download the full file, split it by newlines and pass each item into one of the many JSON parsers.
With these properties, line-delimited JSON streaming is a suitable syntax the web resource.

With the web resource and the smart contract being disconnected, a set of rules has to be defined to ensure interoperability. Unlike in a blockchain, a web resource can change its content as many times as needed. This makes it hard for the users of the blockchain to keep track of updates, and to verify whether the content of the webpage is still the same as it was when the entry into the blockchain was created. To avoid these issues, a new rule is added to the protocol: The contents of a web resource must be immutable and can never be changed. If a customer desires to update the data, a new resource must be created under a different URL and the smart contract needs to be updated with the new pointer.

To prevent customers from modifying the content of their web resources (which they can), a SHA256-hash of the body of the resource needs to be included in the report that gets added to the smart contracts. Clients should generate hashes of the web resources themselves and should reject reports that do not have matching hashes.
This technique is inspired by 'Subresource integrity' \cite{SubresourceIntegrity}, which validates resources like stylesheets and scripts in browsers and is already widely deployed.

Immutability also brings another advantage: If an asset is static, then it never needs to be generated on-the-fly when requesting it and it can be easily deployed to a Content Distribution Network (CDN), which is harder to deny using a DDoS attack.

To enable immutability, the \texttt{Pointer} struct in the smart contract simply contains another property, a \texttt{hash} with a type of \texttt{string}. Accordingly, the \texttt{setPointer()} method gets updated as well.

In variant 1 of the contract, a report object contained:
1. An IP address of the source with optional mask.
2. An IP address of the destination with optional mask.
3. An expiration date.

Therefore, the web resource contains these fields as well. Since multiple reports can be stored under one URL, an array is used and stored under the \texttt{reports} key.

\begin{lstlisting}[
    caption={Example web resource content},
    captionpos=b,
    label={SampleWebResource}
]
{
    "reports": [
        {
            "expirationDate": 1502755200000,
            "sourceIp": {
                "ip": "::ffff:2e65:6095",
                "mask": 120
            },
            "destinationIp": {
                "ip": "::ffff:1234:abcd",
                "mask": 120
            }
        },
        {
            "expirationDate": 1502755200000,
            "sourceIp": {
                "ip": "::ffff:efab:4321",
                "mask": 80
            },
            "destinationIp": {
                "ip": "::ffff:efab:4321",
                "mask": 80
            }
        }
    ]
}
\end{lstlisting}

Instead of using an array, it would also be possible to create two separate JSON reports and delimit them with a newline (\texttt{{\textbackslash}n}) to enable streaming.

In the following, a few rules are established to avoid ambiguity and security vulnerabilities: The timestamps should be UNIX timestamps with miliseconds. IPs should be in IPv6 format (short notations are allowed), masks should be values from 0-128. Clients should check and reject the reports if they are not in the IP boundary set by the smart contract.

In addition to the \texttt{reports} field, other fields are supported for context and metadata:

\begin{description}

\item [version] A string specifying the used version of the protocol to make the protocol future-proof. The versioning should follow Semantic Versioning \cite{Semver}.

\item [whitelist] can either be \texttt{true} or \texttt{false}. The default is \texttt{false}. When this flag is set to \texttt{true}, all reports in the \texttt{reports} array should be considered whitelist entries.

\end{description}

The specification can be expanded in the future by adding more fields and increasing the version number. DOTS \cite{IETFDraft} allows more fields, such as limiting a report to specific port numbers, adding metadata about the attack (duration, attack type, registration time, mitigation status). These information can be considered for addition to the format as well in a future version.

The SHA256 hash of code snippet \ref{SampleWebResource} is \texttt{xZ9hL0AColp7EQ82H/LuAGrGjr5fA60K/vXMjISqnIA=} \footnote{On macOS or Linux a hash can be generated with the following command: \texttt{curl \$RESOURCEURL | openssl dgst -sha256 -binary | openssl enc -base64 -A}}. This value is set for the \texttt{hash} field when registering the web resource in the smart contract.
