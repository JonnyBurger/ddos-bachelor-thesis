\section{Contract 2: Pointer to web resource}
The main disadvantage of storing the reports directly in the contract is that the cost of the data entry scales linearly with the number of reports. What is just a few cents in gas fees for a few reports, can grow expensive at scale.
Ethereum disincentivizes the storage of large data because each node needs to keep track of a whole blockchain by downloading it. Therefore it is also a concern for the Ethereum ecosystem to store large amounts of data.

The second variant of the smart contract works around the space constraints and the big cost of the first variant by storing the list of IP addresses on a web resource and pointing to it on the blockchain. The advantage of this variant is that it works on a much larger scale and is expected to be cheaper in the long-term. The disadvantages are that a web server needs to be set up and a separate specification has to be defined for the format of the web resource. This solution is also more prone to connectivity issues and does not take full advantage of the decentralization and immutability of the blockchain.

\subsection{Smart Contract}

\subsection{Web resource}
When pointing to a web resource, the design of the architecture that is not on the blockchain needs to be considered. By storing a URL in the blockchain, it is already implied that the connection to the resource is made using HTTP(S), which is a suitable protocol for our needs.

The format of the web resource pointed to could be in any format imaginable. A syntax and a data structure has to be settled on. In this section, various aspects of the web resource format are discussed and a format will be developed.

Among the desirable properties that are:

\textbf{1. Portability:} An already common syntax like XML, CSV or JSON is desired because client libraries are already available and no further specifications are needed.

\textbf{2. Upgradeability:} The format of the web resource should be able to be developed further in the future to enable more features, with backwards compatibility. This is more difficult with a two-dimensional design like CSV, because the format can only be extended by adding more rows.

\textbf{3. Streamability:} As one can expect these lists to become very large, it is beneficiary to have the format in a way where it can already be partially evaluated while not yet fully loaded. Downloading a resource fully will take time and it is desirable to not have to load the full list into memory. This is easy with CSV, as the file can be read line by line, but hard with XML or JSON, which needs to be fully loaded in order to be valid.

These properties are conflicting with each other, as no mentioned format supports upgradeability or streamability at the same time.

Additionally, it needs to be decided whether the web resource should be immutable or not. Immutable means that the content of the web page can never change. Additions or deletions to the list are not supported, if the list had to be updated, it would have to move to a different URL and the entry in the blockchain has to be updated as well. This is not only a bad thing, as otherwise the client who accesses the resource has to worry about checking for updates.

A hash can be generated for an immutable file, which could be stored in the blockchain as well and allow the clients to verify that the list has not been altered since it was registered in the smart contract, which is more inline with the rest of architecture that resides in the blockchain. If an asset is immutable, it can be easily outsourced to a CDN, which is harder to deny using a DDoS attack.
In summary, immutability makes it slower and more expensive to register changes, but has the advantage of making it easy to manage changes.
