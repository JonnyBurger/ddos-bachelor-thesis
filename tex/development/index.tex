\chapter{Development}

\section{Solidity Primer}

In the following, three variants of a DDoS attack signaling protocol are being developed. For that, a smart contract is being written in the Soldity programming language \cite{Solc} for each variant. Code-wise, a smart contract strongly resembles a 'class' that is known from object-oriented programming. The following is a 'Hello World' smart contract from the Ethereum website (https://www.ethereum.org/greeter).

\input{snippets/greeter.tex}

Instead of the \texttt{class} keyword, Solidity uses a \texttt{contract} keyword. Inheritance is possible using the \texttt{is} (rather than \texttt{extends} in Java) keyword. A contract can, like a class, be instantiated. The constructor is defined by the method within the contract that has the same name as the contract - in this case, \texttt{function greeter(string greeting)} is the constructor of the \texttt{greeter} contract. Methods can be declared public or private. They are(similar to Javascript) prefixed with the \texttt{function} keyword. A special type in Solidity is the \texttt{address} type, which can hold an 20-byte address of an Ethereum network user.

This Solidity code can be compiled to bytecode and deployed on an Ethereum blockchain using free tools like solc \cite{Solc} and web3 \cite{web3}. When deployed, the contract is stored in a block, alongside with other data that users committed to the Blockchain, and synced to all users of the network. Downloading the complete public Ethereum Blockchain requires dozens of Gigabytes of space \cite{EthereumBlockchainSize}, and it is ever-growing. Once the deployment is finished, Ethereum users can instantiate the smart contract. If they choose to do so, they  send a transaction to the Ethereum network and the instance of the contract is registered on the Blockchain. Methods can also be executed by sending a transaction to the Ethereum Blockchain.

In each method body, a \texttt{msg} global object is available, containing information about the transaction being executed, including \texttt{msg.sender} (the \texttt{address} of the transaction sender), \texttt{msg.gas} and \texttt{msg.value} (for sending Ether currency).

In addition to that, a second global object \texttt{block} gives information about the current block, including \texttt{block.number} and \texttt{block.timestamp}.

A constant method, like \texttt{function greet() constant returns (string)} in the example above is a special function that does not trigger a transaction. Instead, it is a getter function that only executes locally. Constant functions provide convienient interfaces for reading data, but all data should be considered public.

\section{IPv6 considerations}

With IPv4 addresses being 32 bits long, only $ 2^{32} $ combinations are possible and the amount of free addresses is almost exhausted \cite{IPv4Exhaustion}. This leads to the situation that there is currently a transition phase from IPv4 to IPv6. Therefore it makes sense to support both formats.

An IPv4 address can be represented in IPv6 using a format that is defined in RFC 3493 \cite{RFC3493}: 80 bits of zeros, 16 bits of ones, followed by the IPv4 address. For example, the IPv4 address \texttt{46.101.96.149} would be \texttt{0:0:0:0:0:ffff:2e65:6095} in IPv6 hex representation. This notation, most notably, is already supported by the Linux kernel \cite{IPv6Wiki16}. The Google Chrome browser and the Firefox browser will, when entering \texttt{http://[::ffff:2e65:6095]}, display the same website as when entering the IPv4 notation, \texttt{http://46.101.96.149}, which is easily verifiable.

This makes it possible to greatly simplify support for both IPv4 and IPv6, with no flag needed to indicate which version of the protocol is meant. All addresses can be stored in an IPv6 format (using an \texttt{uint128} type) and if the bits 81-96 are all ones, it should be considered an IPv4 address.

\section{Workflow considerations}
Developing a smart contract requires a compiler and a blockchain on which the developer can execute the smart contract. Additional tools can be used to improve development speed, code quality and developer experience.

\subsection{Environment}

The main Ethereum blockchain is unsuitable for validating the correctness of the code manually. There are significant costs to deploy a contract to the main blockchain, also it is not possible for a developer to get immediate feedback because of transaction processing times. It is also insensitive to use the main blockchain for testing purposes, as all network participants need to download all blocks.

The 'Testnet' is a separate Ethereum blockchain for testing purposes. The Ether needed to deploy contracts is not traded on exchanges and was mined with a much lower difficulty level, so usage of the Testnet is basically free. However, the Testnet is also a globally shared blockchain with confirmation times and is not perfect for development. 

\texttt{TestRPC} \cite{TestRPC} is a library that makes it possible to set up a local blockchain for testing purposes. Using \texttt{TestRPC}, it is possible to simulate deployments and transactions of smart contracts with no confirmation delay. Out of the box, \texttt{TestRPC} sets up multiple Ethereum accounts, making it simple to switch the message sender address and test whether the access control features of the developed smart contract are working as intended. This makes \texttt{TestRPC} the ideal blockchain for developing.

\subsection{Testing}

A compiler, such as solc \cite{Solc}, will warn about syntax errors and does not compile invalid Solidity code, indicating to the developer that there is an error in the code.

\texttt{solc} does refuse to compile code that has operations of incompatible types, invalid variable redeclarations, invalid return types and incorrect syntax. 
It does however not give an error for unused variables, dead code, missing arguments and does not completely protect against runtime errors or gas limit errors.

While compilers provide a first layer of assurance by only compiling valid contracts with valid syntax, it is still possible to write code with bugs and security vulnerabilities.

Testing is a technique used in almost all fields of software development to reduce unintended regressions introduced when modifying code. A set of test cases is defined which a testing framework can run through and determine whether all assertions still pass. It is the automation of manual quality assurance.

Cases that can be tested to improve robustness are: Intended smart contract use, malicious smart contract use, and edge cases. For example: Calling a function without permission, calling a function more often than expected, calling a function with unexpected arguments, such as null arguments, wrong types or a big payload.
A robust contract should include unit tests validating that normal use of the contract features result in correct behavior, as well as tests for edge cases and abuse of the contract features.

Usually testing frameworks are written in the language that they are made for testing. There is no testing framework in  Solidity, however there is\texttt{web3.js} \cite{web3}, which exposes a Javascript interface for creating contracts, reading from contracts, and calling transactions. This makes it possible to select from an array of available Javascript testing frameworks. In this section, it is described how to test Solidity code with the \texttt{ava} framework. However, this choice is arbitrary and not important, as testing with another framework will work similarly.

With a macro function, it is possible to create an isolated blockchain for each tdst scenario. The header for each test file is:

\input{snippets/ava-blockchain-each.tex}

The dependencies \texttt{ava}, \texttt{web3}, \texttt{ethereumjs-testrpc} are imported at the top of the file. \footnote{They need to be installed first using the command \texttt{npm install ava web3 ethereumjs-testrpc}, assuming node.js is installed on the computer.}. The function \texttt{makeBlockchain} creates a testing blockchain using \texttt{TestRPC}, and returns an interface for interacting with it, as well as a list of Ethereum account addresses that were generated. 2 accounts were created in this instance, which is sufficient to test from the perspective of a contract owner and an non-provileged user.

For each test, \texttt{makeBlockchain()} is called beforehand and separate blockchain interface and list of addresses is generated and made available. This prevents test cases from interfering with each other.

Consider the 'Hello World' contract from the beginning of this chapter. A simple test would be to create an instance of the contract and test if the \texttt{greet()} function would return the string that was passed to the constructor as the first argument. This test can be implemented in \texttt{ava} with the following code:

\definecolor{dkgreen}{rgb}{0,0.6,0}
\definecolor{gray}{rgb}{0.5,0.5,0.5}
\definecolor{mauve}{rgb}{0.58,0,0.82}
\definecolor{lightgray}{rgb}{.9,.9,.9}
\definecolor{darkgray}{rgb}{.4,.4,.4}
\definecolor{purple}{rgb}{0.65, 0.12, 0.82}

\lstdefinelanguage{JavaScript}{
  keywords={typeof, new, true, false, catch, function, return, null, catch, switch, var, if, in, while, do, else, case, break},
  keywordstyle=\color{blue}\bfseries,
  ndkeywords={class, export, boolean, throw, implements, import, this},
  ndkeywordstyle=\color{darkgray}\bfseries,
  identifierstyle=\color{black},
  sensitive=false,
  comment=[l]{//},
  morecomment=[s]{/*}{*/},
  commentstyle=\color{purple}\ttfamily,
  stringstyle=\color{red}\ttfamily,
  morestring=[b]',
  morestring=[b]"
}

\lstset{
   language=JavaScript,
   aboveskip=3mm,
   belowskip=3mm,
   extendedchars=true,
   basicstyle=\small\ttfamily,
   showstringspaces=false,
   tabsize=3,
   breaklines=true
}


\lstinputlisting{snippets/ava-blockchain-deploy.js}


At first, the contract code gets read from a file and compiled. The compiler, \texttt{solc}, returns the bytecode of the contract, as well as an 'Application Binary Interface' (ABI) in JSON format, which contains information about which methods are available. The ABI is necessary because that information cannot be inferred from the bytecode.

Then, the contract gets instantiated with a 'Hello!' string as the first argument. The address from which this transaction is sent, the bytecode and the amount of gas also has to be specified.
In normal circumstances, a password for the address also has to be provided, but in a \texttt{TestRPC} environment, it can be disabled.
The provided gas in this example is hard-coded for simplicity \textendash{} a more sensible solution would be to estimate gas using the \texttt{estimateGas} helper function provided by \texttt{web3} \footnote{In the project files, a helper function is defined under \texttt{lib/estimate-gas.js}, which handles gas estimation for the whole project.}.

The final argument is a 'callback function'. \texttt{web3} provides a non-blocking asynchronous interface, which means instead getting a return value, a function is called \footnote{By using 'Promises', a syntactic sugar language feature in Javascript, callback functions can be avoided. Promises are used in  the actual project, however in this example classic callback functions are used to avoid confusion.}.
\texttt{web3} oddly calls the callback function twice \textendash{} only after the second time the transaction is committed to the blockchain.

Once the contract instance is created and on the blockchain, methods of the contract can be called. \texttt{myContract.greet()} takes another callback function which is called with the return value of the method. The return value in this case is expected to be 'Hello!'. If this assumption is not true, \texttt{ava} would throw an error here.

When running the test \footnote{Run the test using \texttt{./node\_modules/ava/cli.js test/greeter.js} in the project}, the test framework should give \texttt{1 passed} as the output and exit with code 0.

In addition to testing, critical contracts should be audited by computer security professionals before being deployed. As this thesis does not yet aim to provide a production-ready platform, no external audits were performed.


\section{Contract 1: Native storage}
The first variant stores all reports in the blockchain natively. No optimizations regarding speed and cost are being made, all IP addresses and metadata are simply stored in an array.

All the code in this section is assumed to be in the contract body:
\lstinputlisting{snippets/array-store-shell.sol}

Inside the contract body, two structs are defined, with syntax resembling that of the C programming language:

\lstinputlisting{snippets/structs.sol}

For each IP address, a mask can be specified. This makes it possible to specify a range of IP addresses with no redundancy. 
When discussing masks, the notation of '127.0.0.0/24' is used, where everything before the slash represents the IP address base and the number behind the slash represented the mask. Assuming IPv4, a mask of '/32' means specifically and only that IP, while '/0' means the whole range of IP addresses possible. For example, '127.0.0.0/24' means all IP addresses from 127.0.0.\textbf{0} to 127.0.0.\textbf{255} (all addresses that match the first 24 bits if the IP address base). In IPv4, the maximum value for a mask is 32, in IPv6, the maximum value for a mask is 128.

An entry that can be added to the smart contract is a composite of 3 values: The 'victim IP' or destination IP, the 'attacker IP' or source IP and an expiration date. Expired reports can be filtered by comparing to the \texttt{block.timestamp} global variable.

\clearpage

The next part of the contract is the constructor function.

\lstinputlisting{snippets/array-store-constructor.sol}

The constructor function takes two arguments, an IP address and a mask, which form the 'IP Boundary'. The boundary makes it possible for the creator of the smart contract to restrict the destination IP addresses that can be added to only a certain range. This concept is taken from the original paper \cite{OriginalPaper}.

The address of the creator of the contract instance gets stored in the \texttt{owner} property. This makes it possible to write access control logic in other parts of the contract.

The constructor uses a 'modifier' called \texttt{needsMask}. A modifier is a piece of code that is being run before the method body. The modifier \texttt{needsMask} simply throws when the user calls the constructor without the second argument (which the language itself allows).

If the second argument is missing, the mask would default to 0, encapsulating all IP addresses possible. A user of the smart contract could inadvertently give permission to an user to register reports for the whole range of IP addresses by forgetting a method argument, hence the check using the modifier.

The underscore statement in Solidity can only be used in modifiers. Its effect is that it jumps to the main method body immediately.

The following method is for adding a 'customer'.

\lstinputlisting{snippets/array-store-create-customer.sol}

Only the owner of the contract can add a customer, otherwise the method throws an error. In addition to checking ownership of the contract, the contract also checks if the mask argument was supplied using the previously discussed \texttt{needsMask} modifier.

The method also checks if the supplied IP range is outside the IP boundary and throws if this is the case. For that, if the mask of the IP boundary is \texttt{n}, the last 128 - \texttt{n} bits of both IP addresses are set to 0 and then both IP addresses are compared to each other. For example, to find out if \texttt{::127.0.200.20/120} is in the \texttt{::127.0.0.1/112} boundary, the last 16 bits (128 - 112) are set to zero in both addressed. Then, because \texttt{::127.0.0.0 = ::127.0.0.0}, it is true that the first IP range is included in the second one. An additional check has the be made whether the supplied mask has a smaller numerical value than the mask of the IP boundary. If this is the case, then it is automatically a violation because it cannot be a subset of the IP boundary.

The following method provides an interface for registering a report:

\lstinputlisting{snippets/array-store-block-fn.sol}

Two cases are distinguished: If the owner of the smart contract calls the method, the rule gets applied to the whole IP boundary. Otherwise, the rule gets applied to the range of IP addresses that was registered using the \texttt{createCustomer} method.

The creator of the smart contract can restrict for which IP address ranges the customer can add reports, but the customer can add reports in that range without contacting the smart contract owner. The correct permissions are verified by the other blockchain users who are executing the transaction also and updating their state of the blockchain.

This code is enough to allow for customers adding reports to the contract. Because of a technicality, the reports array can not be marked as \texttt{public}, because public nested structs are not supported in Solidity at the time of writing. For programming with Solidity, it is generally advised to keep the data structure as flat as possible to avoid this problem.
Although all data in a contract is technically public, it is complicated to access, as disassembly of native blockchain data is required. In order to create an interface where blockchain users can read nested structs, it is necessary to flatten the structure into one-dimensional arrays.

\lstinputlisting{snippets/array-store-filter.sol}

The \texttt{blocked} function calls helpers functions which are also declared in the contract.

\lstinputlisting{snippets/array-store-helper-fn.sol}

The helper functions \texttt{filter} and \texttt{isNotExpired} composed together form the \texttt{getUnexpired} function which returns all reports that are not yet expired. No lambda functions or arrays of dynamic size are supported in Solidity. Therefore, two for-loops are needed, the first to determine the size of the array that should be created, and the second one to fill an array of that size. This does not make the contract more expensive to operate, since all the methods are marked as \texttt{internal} and are only invoked locally.

The contract now has all methods needed to write and read reports. These methods can be called programmatically using a client library, like \texttt{geth} (the Go client) or \texttt{web3.js} (the Javascript client), or using a graphical interface like Mist \cite{Mist}.

For inserting IPv6 addresses into the contract from a client interface, it needs to be converted into a 128-bit integer. Helper libraries exist for this task, for example the \texttt{ip-address} package on the npm (Node package manager) registry \cite{IpAddressNpm} allows to easily convert a string representation of an IP address to a big integer and vice versa:

\lstinputlisting{snippets/ip-address-npm.js}

\section{Contract 2: Pointer to web resource}
The main disadvantage of storing the reports directly in the contract is that the cost of the data entry scales linearly with the number of reports. What is just a few cents in gas fees for a few reports, can grow expensive at scale.
Ethereum disincentivizes the storage of large data because each node needs to keep track of a whole blockchain by downloading it. Therefore it is also a concern for the Ethereum ecosystem to store large amounts of data.

The second variant of the smart contract works around the space constraints and the big cost of the first variant by storing the list of IP addresses on a web resource and pointing to it on the blockchain. The advantage of this variant is that it works on a much larger scale and is expected to be cheaper in the long-term. The disadvantages are that a web server needs to be set up and a separate specification has to be defined for the format of the web resource. This solution is also more prone to connectivity issues and does not take full advantage of the decentralization and immutability of the blockchain.

\subsection{Smart Contract}

\subsection{Web resource}
When pointing to a web resource, the design of the architecture that is not on the blockchain needs to be considered. By storing a URL in the blockchain, it is already implied that the connection to the resource is made using HTTP(S), which is a suitable protocol for our needs.

The format of the web resource pointed to could be in any format imaginable. A syntax and a data structure has to be settled on. In this section, various aspects of the web resource format are discussed and a format will be developed.

Among the desirable properties that are:

\textbf{1. Portability:} An already common syntax like XML, CSV or JSON is desired because client libraries are already available and no further specifications are needed.

\textbf{2. Upgradeability:} The format of the web resource should be able to be developed further in the future to enable more features, with backwards compatibility. This is more difficult with a two-dimensional design like CSV, because the format can only be extended by adding more rows.

\textbf{3. Streamability:} As one can expect these lists to become very large, it is beneficiary to have the format in a way where it can already be partially evaluated while not yet fully loaded. Downloading a resource fully will take time and it is desirable to not have to load the full list into memory. This is easy with CSV, as the file can be read line by line, but hard with XML or JSON, which needs to be fully loaded in order to be valid.

These properties are conflicting with each other, as no mentioned format supports upgradeability or streamability at the same time.

Additionally, it needs to be decided whether the web resource should be immutable or not. Immutable means that the content of the web page can never change. Additions or deletions to the list are not supported, if the list had to be updated, it would have to move to a different URL and the entry in the blockchain has to be updated as well. This is not only a bad thing, as otherwise the client who accesses the resource has to worry about checking for updates.

A hash can be generated for an immutable file, which could be stored in the blockchain as well and allow the clients to verify that the list has not been altered since it was registered in the smart contract, which is more inline with the rest of architecture that resides in the blockchain. If an asset is immutable, it can be easily outsourced to a CDN, which is harder to deny using a DDoS attack.
In summary, immutability makes it slower and more expensive to register changes, but has the advantage of making it easy to manage changes.

\section{Contract 3: Bloom filter}

While variant 1 of the smart contract is space-inefficient and variant 2 needs additional infrastructure, variant 3 is an attempt to strike a balance. It is a standalone contract that solves the scalability issue by using a bloom filter.

A bloom filter is a probabilistic data structure that is very space efficient. In the context of large amounts of IP addresses, it can be tested if an IP address has been inserted into the bloom filter beforehand with constant space requirements. False positives are possible, false negatives are not.

This variant is space-efficient and the logic is self-contained, however perfect accuracy cannot be guaranteed anymore. Also, additional parameters like a blacklist/whitelist flag, expiration dates, ip boundaries can not be stored in a bloom filter, as it is impossible to fully retrieve the information that the bloom filter was fed.

It is however possible to store additional information in the contract, and use multiple contracts if the parameters differ for reports.

No implementation of a bloom filter in Solidity could be found online, therefore the approach taken to implement this variant was to convert one of the countless examples written in other programming languages into Solidity.

One part of a bloom filter is the hash function that takes any string and converts it into a fixed-length hash. The other part is managing the store, and exposing interfaces for adding entries and checking if an entry has been added.

Different hash functions are available and the choice of the hash function influences the properties of the bloom filter, mainly accuracy and speed. A bloom filter becomes more accurate if the hash function it uses produces more uniform hashes. However, usually more uniform results also means slower hashing.

\subsection{Hashing function}

The hashing part is the more complex code to convert to Solidity, since hashing functions use big numbers and bit-shifting to generate their hashes, which both are very prone to differences in different environments.

In an first attempt, the popular hashing library \texttt{imurmurhash} \cite{imurmur} was taken and rewritten in Solidity as close as possible. This library is using the \texttt{\textgreater{}\textgreater{}\textgreater} (logical right shift) operator which is absent in Solidity. By using a \texttt{\textgreater{}\textgreater} (arithmetic right shift) operator instead, the hashes could not be reproduced. It also generated big numbers in the process whose behavor was inconsistent between Ethereum VM and Javascript environments.
When later using the ported hash function in a bloom filter, it would return many false positives\footnote{A test case showing the false positive is available under \texttt{test/hash-function.js} in the project files.}.

In an second attempt, the \texttt{bloomfilter.js} \cite{bloomfilterjs} library was ported to Solidity as close as possible. This library uses the Fowler-Noll-Vo hash function instead. The commit history shows that it had once worked without the \texttt{\textgreater{}\textgreater{}\textgreater} operator, but that operator was added later. Yet, the Solidity implementation did still not yield the same hashes for the same input as the Javascript equivalent of the code. By testing subfunctions of the hash function, it was discovered that the inconsistency is caused by bug numbers. In one hash operation, the statement \texttt{2166136261 \textasciicircum{} 65 \& 0xff} would be executed. It is easily verifiable that this expression evaluates to \texttt{-2128831100} in Javascript and to \texttt{2166136196} in Solidity Version 0.4.9.

This difference results because Solidity uses types for numbers such as \texttt{int}, while Javascript only supports floating-point math. With the different hashes being generated, the contract that resulted from the second attempt also led to false positives. \footnote{A test case showing the false positive is available under \texttt{test/hash-function-2.js} in the project files.}

Porting existing hash functions to Solidity provides great challenges. For the third attempt, bloom filter implementations were searched that used one of the hash functions already implemented in Solidity. Two hashing algorithm functions are globally available in every Solidity contract, \texttt{sha256()} and \texttt{sha3()}. For the third attempt, a 'Simple Bloom Filter' \cite{SimpleBloomFilter} code snippet from Github was used as the basis for the smart contract.

This third attempt proved successful as the SHA-256 hashes are reproducible and false positives did not occur anymore. The smart contract that resulted from this attempt has also the most concise and easiest to read code.

\subsection{State management and interface}

A bloom filter creates a fixed length array that initially only contains zeroes. The result of the hashing of the input determines which array items are being switched from zero to one or vice versa. The size of the array is variable.

Instead of only hashing an input value once, it usually is hashed multiple times. This reduces the chance of collision \cite{MultipleHashes}. The amount of hash iterations is also variable. In this simple bloom filter, additional hashes beyond the first one are just bit-shifts of the first hash.

The bigger the amount of the array items and the bigger the amount of the hash iterations, the more accurate a bloom filter gets. The default parameters of the Python bloom filter were an array size of 1024 and 13 hash iterations. \cite{SimpleBloomFilter} 

In Solidity however, this configuration would raise an 'out of gas' error when adding a string to the filter. 
The maximum amount of gas (also called gas limit) that can be spent on a transaction is 3,141,592 (pi million). Since every node in the network needs to download every transaction, there is an artificial cap on how computationally expensive a transation may be.

By decreasing the array size to 512, the transaction stays below the gas limit and the bloom filter works. 
By testing different values, it was determined that an array size of 836 is the maximum before this specific contract would not be able to have a transaction executed.

The amount of hash iterations would not significantly affect gas cost. This is because bit-shifts are relatively cheap, and because Ethereum weighs storage more heavily than computing power when it comes to cost calculation.

\subsection{IPv6 format ambiguity consideration}

A IPv6 address can be formatted in more than one way. \cite{RFC4291I66} For instance, leading zeroes in one block can, but don't have to, be omitted. Also, multiple subsequent blocks containing only zeroes can, but don't have to be replaced by two colons.
For example, \texttt{0000:0000:0000:0000:0000:ffff:2e65:6095} can also be formatted as
\path{0:0:0:0:0:ffff:2e65:6095} or even as \texttt{::ffff:2e65:6095}. Passing to the hash function the same IP address, but in different formats, will result in vastly different hashes, assuming an uniform hash function.

To prevent false negatives, the possibility of multiple representations per IP addresses has to be eliminated. Therefore, it makes sense to forbid shorthand notations as well as IPv4-in-IPv6 notations such as \texttt{::ffff:46.101.96.149} and any other alternative notations that RFC 4291 \cite{RFC4291I66} mentions.
This does not result in higher storage costs, as a SHA256 hash is always only 256 bits long.



\section{IP address ownership verification}

In all variants of the smart contract, owners of destination IP addresses can store source IP addresses they want to be blocked in the smart contract. In order to establish trust, it was suggested by the original paper \cite{OriginalPaper} there should be a way to automatically verify the ownership of the destination IP addresses, leaving the implementation open.

This problem was explored during the development of the prototype, however it turns out to be a challenging task. Using certificates to validate IP address ownership in Solidity is currently not practical for at least two reasons.

The main issue is obtaining a certificate. As there is no way to directly mathematically or logically prove that somebody is the owner of an IP address (it can be spoofed \cite{IPSource38}), an indirect solution is required.

Theoretically, the certificate process of domains could be applied to IP addresses. Certificate Authorities (CA) are institutions whose business is to issue and manage certificates. They verify and validate the ownership of domains and issue certificates for it. CAs need to fulfill a wide range of requirements \cite{BaselineRequirements} to be considered reputable and be included in the root key store of operating system and browser vendors. Currently, only 156 certificates from 60 different owners are trusted in Firefox \cite{httpscca67}.

To fulfill the strict requirements, CAs need to make investments in establishing a system that securely validates a domain and manages the certificates that are issued. To make money, nearly all CAs that are trusted in Firefox charge an issuance fee for domains. The notable exception is "Let's Encrypt" \cite{LetsEncrypt}, which makes money through sponsorships.

Although it is technically possible to issue an SSL certificate for an IP address, it is very uncommon. GlobalSign \cite{GlobalSign} is the only provider whose certificates are trusted by Firefox and that issues certificates for IP addresses. To obtain a certificate, it is a requirement that the IP address is registered in the RIPE database \cite{Database95}, which most are not, and a certificate starts at \$349.

Concluding the overview of the certificate issuance process, there are no suitable providers offering certificates for IP addresses and it is an expensive endeavor to build a certificate authority whose complexity quickly becomes bigger than the one of the scope of the thesis.

Assuming it would be possible to obtain and validate certificates for IP addresses, it is a computationally expensive task that would likely reach the gas limit of Ethereum. This is just a hypothesis however, as there is no implementation of certificate verification in Solidity. Certificate verification, as it is most commonly done with OpenSSL, would have to be ported ported to Solidity, which is complex.
There is however a proposal to add certificate validation on a language-level \cite{EIP74}. As of writing, the exact implementation is not clear, with parts of the community vouching for direct RSA signature verification, and other parts wanting BigInt Support which could then enable certificate validation.

Certificate validation in Solidity is not a hard requirement though \textemdash{} everything in the blockchain is public including stored certificates and IP addresses, it can also be done off the blockchain.

To propose an alternative solution for identity verification, each Ethereum transaction is signed by the user and the \texttt{msg.sender} value is verified by the network, guaranteeing the authenticity of the sender address. By pre-validating the destination IP addresses before a customer is added, no reports can be added to the contract on others behalf.


\section{Security considerations with Solidity}

Applications written in Solidity deserve special security considerations for several reasons.
Unlike classical applications, code for transactions runs on every node in the network and any user can call any method. Therefore it is necessary to implement proper access control for each method.

The source code of smart contracts usually is made public after the deployment of it to allow users to verify the behavior of the contract before they interact with it. This increases the chance of bugs being found. Examples of it are TheDAO contract and the Parity multisig wallet, both of which were hacked because a vulnerability was found in the source code \cite{DAO} \cite{MultisigHack}.

All data stored in the smart contract has to be considered public. A rock-paper-scissor contract which people used for gambling turned out to have a trivial flaw where the first move could be extracted from the blockchain - rock would have the value \texttt{0x60689557} and scissor and paper would have a different one \cite{ThinkingAboutSmartContractSecurity}.

The main takeaway for the contracts described in this contract is that the stored IP addresses will be public (although maybe obfuscated) to everybody. Even attackers can determine the list of IPs that are blocked if they know the contract address.

The creator of Solidity, Christian Reitwiessner, advocates for implementing a 'fail safe' mode that can be activated in case of a hack that will turn the contract into a read-only, withdraw-only mode \cite{FailsafeMode}.

The creator of Ethereum, Vitalik Buterin, has compiled a list of vulnerabilities based on real-world exploits. To be safe against the most common vulnerabilities, the following practices should be followed \cite{ThinkingAboutSmartContractSecurity}:

\begin{description}
    \item[Constructor naming] The constructor needs to have the same name as the contract. If by mistake, the constructor has a different name, it is not called on deployment, but can be called as a transaction. In many contracts, including the variants in this thesis, calling the constructor makes the caller the owner of the contract. The 'Rubixi' contract suffered from this bug and led to a takeover vulnerability \cite{ThinkingAboutSmartContractSecurity}.
    \item[Loops] Loops are susceptible to gas limit failures and can be stalled. Therefore, the number of iterations in a loop should not be controlled by transaction parameters. Additionally, the \texttt{var} keyword should not be used within a \texttt{for} statement, as \texttt{var} is interpreted as \texttt{uint8}, which would lead to an overflow if the loop exceeds 256 iterations.
    \item[Call stack depth] The 'call stack depth' describes the amount of nested function calls \textemdash\ it increases when a function calls another function, and if a function returns, the call stack depth decreases. A long call stack can be produced by excessive recursion. Solidity has a 1024 call stack depth limit \textemdash\  this means that Solidity could for example not compute the 1025th fibonacci number using recursion. This limit is an attack vector. An attacker could craft a function that calls itself 1023 times and on the 1024th time calls another, vulnerable function, that stops execution as soon as a subfunction is called, because the maximum call stack size is reached. Therefore contracts must be designed to not expose vulnerabilities when a function that is susceptible to a call stack depth attack gets only partially executed. The core development team of Ethereum proposes to solve this problem on a language-level in EIP \#150 \cite{EIP150}.

\end{description}
