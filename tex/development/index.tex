\chapter{Development}

In the following, three variants of a DDoS attack signaling protocol are being developed. For that, a smart contract is being written in the Soldity programming language \cite{Solc} for each variant. A smart contract strongly resembles a 'class' that is known from object-oriented programming. The following is a 'Hello World' smart contract from the Ethereum website (https://www.ethereum.org/greeter).

\definecolor{dkgreen}{rgb}{0,0.6,0}
\definecolor{gray}{rgb}{0.5,0.5,0.5}
\definecolor{mauve}{rgb}{0.58,0,0.82}

\lstset{frame=tb,
  language=Java,
  aboveskip=3mm,
  belowskip=3mm,
  showstringspaces=false,
  columns=flexible,
  basicstyle={\small\ttfamily},
  numbers=none,
  numberstyle=\tiny\color{gray},
  keywordstyle=\color{blue},
  commentstyle=\color{dkgreen},
  stringstyle=\color{mauve},
  breaklines=true,
  breakatwhitespace=true,
  tabsize=3
}

\lstinputlisting{snippets/greeter.sol}


Instead of the \texttt{class} keyword, solidity uses a \texttt{contract} keyword. Inheritance is possible using the \texttt{is} (rather than \texttt{extends} in Java) keyword. A contract can, like a class, be instantiated. The constructor is defined by the method within the contract that has the same name as the contract - in this case, \texttt{function greeter(string greeting)} is the constructor of \texttt{greeter}. Methods can be declared public or private. They are, like in Javascript, prefixed with the \texttt{function} keyword. A special type in Solidity is the \texttt{address} type, which can hold an 40-byte address of an Ethereum network user.

This Solidity code can be compiled to bytecode and deployed on a Ethereum blockchain. When deployed, the contract is stored in a block, alongside with other data that users committed to the Blockchain, and synced to all users of the network (downloading the complete public Ethereum Blockchain uses dozens of Gigabytes). Once the deployment is finished, Ethereum users can instantiate the smart contract. If they choose to do so, they  send a transaction to the Ethereum network and the instance of the contract is registered on the Blockchain. Methods can also be executed by sending a transaction to the Ethereum Blockchain.

In each method body, a \texttt{msg} global object is available, containing information about the transaction being executed, including \texttt{msg.sender} (an \texttt{address}), \texttt{msg.gas}, \texttt{msg.value} (for sending Ether).

In addition to that, a second global variable \texttt{block} gives information about the current block, including \texttt{block.number} and \texttt{block.timestamp}.

A constant method, like \texttt{function greet() constant returns (string)} is a special function that does not trigger a transaction. Instead, it is a getter function that only executes locally. Constant functions provide convienient interfaces for reading data, but all data should be considered public.

\section{IPv6 considerations}

With IPv4 addresses being 32 bits long, only $ 2^{32} $ combinations possible and the amount of free addresses is almost exhausted \cite{IPv4Exhaustion}, there currently is a transition phase from IPv4 to IPv6. Therefore it makes sense to support both formats.

An IPv4 address can be represented in IPv6 using a format that is defined RFC 3493: 80 bits of zeros, 16 bits of ones, followed the IPv4 address. For example, the IPv4 address \texttt{46.101.96.149} would be \texttt{0:0:0:0:0:ffff:2e65:6095} in IPv6 hex representation. This notation is supported by the Linux kernel and macOS natively, in a web browser \texttt{http://[::ffff:2e65:6095]} will display the same website as \texttt{http://46.101.96.149}.

This makes it possible to greatly simplify support for both IPv4 and IPv6, with no flag needed to indicate which version of the protocol is meant. All addresses are stored in a IPv6 format (using a \texttt{uint128} type) and if the bits 81-96 are all ones, it indicates that it is an IPv4 address.

\section{Contract 1: Native storage}
The first variant stores all IP addresses in the blockchain natively. No optimizations regarding speed and cost are being made, all IP addresses are simply stored in an array.

\lstinputlisting{snippets/array-store-shell.sol}

Inside the contract body, some structs are defined, with syntax resembling that of the C programming language:

\lstinputlisting{snippets/structs.sol}

For each IP address, a mask can be added. This makes it possible to specify a range of IP addresses with no redundancy. 
When discussing masks, a notation such as '127.0.0.0/24' is used, where everything before the slash is the IP address base and the number behind the slash is the mask. A mask of '/32' means specifically and only that IP, while '/0' means the whole range of IP addresses possible. '127.0.0.0/24' means all IP addresses from 127.0.0.0 to 127.0.0.255. Since the contract works with IPv6 addresses, the maximum value for mask is 128.

An entry that can be added to the smart contract is a composite of 3 values: The 'victim IP' or destination IP, the 'attacker IP' or source IP and an expiration date. Expired reports can be filtered by comparing to the \texttt{block.timestamp} global.

The next step is to write the constructor function.

\lstinputlisting{snippets/array-store-constructor.sol}

The constructor function takes two arguments, an IP address and a mask, which form the 'IP Boundary'. The boundary makes it possible for the creator of the smart contract to restrict the destination IP addresses that can be added to only a certain range.

The address of the creator of the instance gets saved in the \texttt{owner} property. This allows to define that the owner of the contract can call more methods than other users.

The constructor also has a 'modifier'. A modifier is piece of code that is being run before the method body. The modifier \texttt{needsMask} simply throws when the user calls the constructor without the second argument (which the language itself allows). The underscore statement in Solidity can only be used in modifiers and its effect is that it jumps to the main method body immediately.

In this contract, there is a method for adding a 'customer'.

\lstinputlisting{snippets/array-store-create-customer.sol}

Only the owner of the contract can add a customer, otherwise the method throws. In addition to checking ownership of the contract, the contract also checks if the mask argument was supplied using the previously discussed \texttt{needsMask} modifier.

Also, the method checks if the supplied IP range is outside the IP boundary and throws if this is the case. For that, if the IP boundary is \texttt{n}, the last 128 - \texttt{n} digits of both IP addresses are set to 0 and they should match up. For example, to find out if \texttt{::127.0.200.20/120} is in the \texttt{::127.0.0.1/112} boundary, the last 16 bits (128 - 112) are set to zero in both addressed. Then, because \texttt{::127.0.0.0 = ::127.0.0.0}, it is true that the first IP range is included in the second one.

The following method provides an interface for registering a report:

\lstinputlisting{snippets/array-store-block-fn.sol}

Two cases are distinguished: If the owner of the smart contract calls the method, the rule gets applied to the whole IP boundary. Otherwise, the rule gets applied to the range of IP addresses that were registered using the \texttt{createCustomer} method. So the creator of the smart contract can restrict for which IP address ranges the customer can add reports, but the customer can add reports in that range without contacting the smart contract owner. The correct permissions are verified by the other blockchain users who are executing the transaction also and updating their state of the blockchain.

This code is enough to allow for customers adding reports to the contract. Because of technicalities, the reports can not be marked as \texttt{public}, because public nested structs are not supported in Solidity. It is generally advised to keep the data structure as flat as possible to avoid this problem.
All data in a contract is technically public, but in machine code that has to be deassembled. In order to create an interface where blockchain users can read nested structs, it is necessary to flatten the structure into plain arrays.

\lstinputlisting{snippets/array-store-filter.sol}

This code calls helpers functions which are also declared in the contract.

\lstinputlisting{snippets/array-store-helper-fn.sol}

The helper functions \texttt{filter} and \texttt{isNotExpired} composed together form the \texttt{getUnexpired} function which returns all reports that are not yet expired. Passing functions to other functions is possible in Solidity, but the filtering is not as straight-forward as it is in most languages. There is not native \texttt{filter} function like in Python or Javascript, and no arrays of dynamic size can be created inside a function body. Therefore, two for-loops are needed, the first to determine the size of the array that should be created, and the second one to fill an array of that size. This does not make the contract more expensive to operate, since all these methods are marked as \texttt{internal} and are only run locally.

The contract now has all methods needed to write and read reports. These methods can be called programmatically using a client library, like \texttt{geth} (the Go client) or \texttt{the Javascript client}. For inserting IPv6 addresses into the contract from a client interface, it needs to be converted into a 128-bit integer. Helper libraries exist for this task, for example the \texttt{ip-address} package on the npm (Node package manager) registry allows to easily convert a string representation of an IP address to a big integer:

\lstinputlisting{snippets/ip-address-npm.js}

\section{Contract 2: Pointer to web resource}
The main disadvantage of storing the reports directly in the contract is that the cost of the data entry scales linearly with the number of reports. What is just a few cents in gas fees for a few reports, can grow expensive at scale.
Ethereum disincentivizes the storage of large data because each node needs to keep track of a whole blockchain by downloading it. 

The second variant of the smart contract works around the space constraints and the big cost of the first variant by storing the list of IP addresses on a web resource and pointing to it on the blockchain. The advantage of this variant is that it works on a much larger scale and is expected to be cheaper in the long-term. The disadvantages are that a web server needs to be set up and a separate specification has to be defined for the format of the web resource. This solution is also prone to connectivity issues and does not take full advantage of the decentralization and immutability of the blockchain.

\subsection{Smart Contract}
The basic data structure of the smart contract is similar to the array store contract. The difference is that the items do not contain IP addresses, but URLs which point to lists of IP addresses. Since it is possible to include as many reports as desired in a web resource, the assumption is made that only 1 pointer is needed per customer. This assumption can simplify the contract, so that no array is needed and that is not necessary anymore to include a helper function that flattens the data.

\lstinputlisting{snippets/ip-pointer-store-outline.sol}

Instead of using an array to store the pointers, a \texttt{mapping(address => Pointer)} is being used. It is the equivalent of a \texttt{Map<address, Pointer>} in Java with it having types, but it uses a Javascript object like syntax for getting, setting and deleting values.

With this design, each user of the smart contract can have one pointer at a time. With each transaction the previous pointer is overwritten, hence the method name \texttt{setPointer}.

The struct \texttt{Pointer} does not nest another struct \texttt{IPAddress} like in the previous contract, but instead stores IP address base and mask directly. Although it would be a cleaner design, it would trigger an error message \texttt{Internal type is not allowed for public state variables}. The reason for this that each Ethereum contract has a JSON interface called Application Binary Interface (ABI) in which a structure like this cannot be represented at the moment, therefore this type of structure is not supported.

For this reason, the contract is programmed to do without nested structs, but with mappings instead. The benefit of mappings is that no helper methods are needed to retrieve the data.
Verifying that the sender of the transaction is a member works by comparing \texttt{members[msg.sender].ip} to 0. In many other languages, \texttt{members[msg.sender]} would be compared to \texttt{null} and the comparison the above code would be susceptible to a null-pointer exception. However, in Solidity, there is no concept of 'null'. Accessing a primitive value that has not been set returns zero, accessing a struct that has not been set returns a struct where all values are zero.

\subsection{Web resource}
Several design aspects need to be considered for the web resource. By storing a URL in the blockchain, it is already implied that the connection to the resource is made using HTTP(S), which is a suitable protocol for our needs.

A syntax and a data structure that the reports are represented in needs to be selected. It is desired to use an already common syntax like XML, CSV or JSON, because there are a selection of clients for many languages available and because of the heavy standardization. The use of an established format increases the \textbf{Portability} of the protocol. An essay by Nicolas Seriot \cite{ParsingJSON} shows the challenges of covering edge cases in standards by showing inconsistencies in trailing commas, unclosed structures, duplicate keys and white space in JSON, making a case against developing an additional format.

JSON and XML allow to extend the schema by adding more keys to the object or by adding more allowed tags to the schema. \textbf{Upgradeability} is also a desired property of the protocol, because it allows the web resource format to be developed further in the future to enable more features with backwards compatibility. This is more difficult with a two-dimensional design like CSV, because the format can only be extended by adding more columns.

The expectation is that lists can become very large, therefore it is beneficiary to have the format in a way where it can already be partially evaluated while not yet fully loaded. This is called \textbf{Streaming}. Network latency and throughput rates do influence the speed in which the files will be downloaded.
Streaming in CSV is simple: as soon as a newline character is detected, the client can safely assume that an entry has been fully loaded and that it can be processed. On the contrary, with XML or JSON, who have closing tags, streaming is not possible.

Neither CSV, JSON or XML have both good Streamability and Upgradeability, and other formats don't have good Portability.

Line delimited JSON streaming \cite{LineDelimitedJSON} provides a reasonable tradeoff. It's syntax is a list of items that are delimited using a newline character (\texttt{{\textbackslash}n}), where each item has a valid JSON object according to RFC 7158 \cite{RFC7158}. Packages for line-delimited JSON streaming exist for at least node.js and Python in their respective registries, so client libraries are available. As a fallback, it is always possible to download the full file, split it by newlines and pass each item into one of the many JSON parsers.
With these properties, line-delimited JSON streaming is a suitable syntax the web resource.

With the web resource and the smart contract being disconnected, a set of rules has to be defined to ensure interoperability. Unlike in a blockchain, a web resource can change its content as many times as needed. This makes it hard for the users of the blockchain to keep track of updates, and to verify whether the content of the webpage is still the same as it was the entry into the blockchain was created. To mitigate these issues, a new rule is added to the protocol: The contents of a web resource must be immutable and can never be changed. If a customer desires to update the data, a new resource must be created under a different URL and the smart contract needs to be updated with the new pointer.

To prevent customers from modifying the content of their web resources (which they can), a SHA256-hash of the body of the resource needs to be included in the report that gets added to the smart contracts. Clients should generate hashes of the web resources themselved and should reject reports that do not have matching hashes.
This technique is inspired by 'Subresource integrity' \cite{SubresourceIntegrity}, which validates resources like stylesheets and scripts in browsers and is therefore already widely deployed.

Immutability also brings another advantage: If an asset is static, it can be easily deployed to a CDN, which is harder to deny using a DDoS attack.

To enable this feature, the \texttt{Pointer} struct in the smart contract simply contains another property, a \texttt{hash} with a type of \texttt{bytes32}. Accordingly, the \texttt{setPointer()} method gets updated as well.

In variant 1 of the contract, a report object contained 1. an IP address of the source with optional mask 2. an IP address of the destination with optional mask 3. an expiration date.

Therefore, the web resource contains these fields as well. Since multiple reports can be stored under one URL, an array is used and stored under the \texttt{reports} key.

\begin{lstlisting}[
    caption={Example web resource content},
    captionpos=b,
    label={SampleWebResource}
]
{
    "reports": [
        {
            "expirationDate": 1502755200000,
            "sourceIp": {
                "ip": "::ffff:2e65:6095",
                "mask": 120
            },
            "destinationIp": {
                "ip": "::ffff:1234:abcd",
                "mask": 120
            }
        },
        {
            "expirationDate": 1502755200000,
            "sourceIp": {
                "ip": "::ffff:efab:4321",
                "mask": 80
            },
            "destinationIp": {
                "ip": "::ffff:efab:4321",
                "mask": 80
            }
        }
    ]
}
\end{lstlisting}

Instead of using an array, it would also be possible to create two separate JSON reports and delimit them with a newline (\texttt{{\textbackslash}n}) to enable streaming.

The timestamps should be UNIX timestamps with miliseconds. IPs should be in IPv6 format (short notations are allowed), masks should be values from 0-128. Clients should check and reject the reports if they are not in the IP boundary set by the smart contract.

In addition to the \texttt{reports} field, other fields are supported for context and metadata:

\begin{description}

\item [version] A string specifying the used version of the protocol to make the protocol future-proof. The versioning should follow Semantic Versioning \cite{Semver}.

\item [whitelist] can either be \texttt{true} or \texttt{false}. The default is \texttt{false}. When this flag is set, all reports in the \texttt{reports} array should be considered whitelist entries.

\end{description}

The specification can be expanded in the future by adding more fields and increasing the version number. DOTS \cite{IETFDraft} allows more fields, such as limiting a report to specific port numbers, adding metadata about the attack (duration, attack type, registration time, mitigation status). These information can be considered for addition to the format as well in a future version.

The SHA256 hash of code snippet \ref{SampleWebResource} is \texttt{xZ9hL0AColp7EQ82H/LuAGrGjr5fA60K/vXMjISqnIA=} \footnote{On macOS or Linux a hash can be generated with the following command: \texttt{curl \$RESOURCEURL | openssl dgst -sha256 -binary | openssl enc -base64 -A}}. This value is set for the \texttt{hash} field when registering the web resource in the smart contract.


\section{Contract 3: Bloom filter}

The third variant of the smart contract is a standalone contract that solves the scalability issue by using a bloom filter.

A bloom filter is a probabilistic data structure that is very space efficient. In the context of large amounts of IP addresses, it can be tested if an IP address has been inserted into the bloom filter beforehand with constant space requirements. False positives are possible, false negatives are not.

The bloom filter contract is a balance between leveraging the blockchain and using little space, however perfect accuracy cannot be guaranteed anymore.

\section{IP address ownership verification}
In all variants of the smart contract, owners of destination IP addresses can store source IP addresses they want to be blocked in the smart contract. In order to establish trust, there should be a way to automatically verify the ownership of the destination IP addresses. This is a challenging task, as is explained in this section. Using certificates to validate IP ownerships in Solidity is currently not practical for at least two reasons.

It is a computationally expensive task that would require more gas than the gas limit allows and there is no implementation of certificate verification in Solidity. Certificate verification, as it is most commonly done with OpenSSL, would have to be ported ported to Solidity, which is complex.
There is however a proposal to add certificate validation on a language-level \cite{EIP74}. As of writing, the exact implementation is not clear, with parts of the community vouching for direct RSA signature verification, and other parts wanting BigInt Support which could then enable certificate validation.

However, since everything in the blockchain is public including stored certificates and IP addresses, it is not required that certificates are validated in Solidity, it can also be done off the blockchain. 

The second challenge is obtaining a certificate. As there is no way to directly mathematically or logically prove that somebody is the owner of an IP (it can be spoofed), an indirect solution is required. It was considered that the certificate process of domains could be applied to IPs: There are Certificate Authorities (CA) who verify and validate the ownership of a domain.

CAs need to make investments in establishing a system that securely validates a domain and manages the certificates that are issued. CAs need to fulfill a wide range of requirements \cite{BaselineRequirements} to be considered reputable and be included in the root key store of operating system and browser vendors. Nearly all CAs for domains issue certificates for a fee or need to generate revenue by sponsorships.

Although it is technically possible to issue an SSL certificate for an IP address, it is very uncommon. GlobalSign \cite{GlobalSign} is the only provider known to us that issues certificates for IP addresses, and requires that the IP is registered in the RIPE database.

The same system that currently exists for domain owner verification could be introduced by the industry for IP addresses verification, but is very complicated and defeats the purpose of using the blockchain as the original idea was to remove the need for additional infrastructure.

\section{Validating Smart contracts during development}
Developing a smart contract requires a compiler and a blockchain on which the developer can execute the smart contract. A compiler, such as solc \cite{Solc}, will warn about syntax errors and does not compile invalid Solidity code, indicating to the developer that there is an error in the code.

While compilers provide a first layer of assurance by only compiling valid contracts with valid syntax, it is still possible to write code with bugs and security vulnerabilities. As with any software, developers need a workflow which allows them to test their changes fast and efficiently.

The main Ethereum blockchain is unsuitable for validating the correctness of the code manually. There are significant costs to deploy a contract to the blockchain, also it is not capable of giving the developer immediate feedback because of transaction processing times. It is also frowned upon to use the main blockchain for testing purposes, as all network participants need to download all blocks.

This is not the case in the Testnet, which is a separate blockchain for testing purposes. Still, the Testnet is a globally shared blockchain and is not perfect for development. Instead, a test framework called \texttt{TestRPC} that makes it possible to set up a local blockchain for testing is used. Using \texttt{TestRPC}, it is possible to simulate deployments and transactions of smart contracts with no confirmation delay. Out of the box, \texttt{TestRPC} sets up multiple Ethereum accounts, making it simple to switch the message sender address and test whether the access control features of the developed smart contract are working as desired. This makes \texttt{TestRPC} the ideal blockchain for developing.

In addition to manual testing, the robustness should be validated by unit tests, just like in any other piece of software. This allows to automatically validate that all assertions are still true when a change is introduced to the contract. A robust contract should include unit tests validating that normal use of the contract features result in correct behavior, as well as tests for edge cases and abuse of the contract features.

Examples of cases that should be tested are: Calling a function without permission, calling a function more often than expected, calling a function with unexpected arguments, such as null arguments, wrong types or a big payload.

\texttt{TestRPC}, the \texttt{solc-js} compiler, and the \texttt{ava} testing framework to set up a test environment. Each test is completely atomic and independent from other ones, that means that for each test, a separate \texttt{TestRPC} blockchain is created and the scenario is run on it. After the test is finished, the blockchain and the accounts get destroyed.

The complete isolation of each test is good practice, as side-effects can be ruled out. With \texttt{ava}, the tests can be run in parallel, making it faster to run the whole test suite. However, there are many other generic test frameworks that work just as well and have their own advantages and disadvantages.

One interesting specialized framework is Truffle \cite{Truffle}, which provides helpers for developing and testing Solidity contracts. Truffle attempts to make the process described in this section easier and looks very promising, but is still in beta. For the reason that the framework is still very much in development and changing, a generic testing framework was used.

For real-world smart contract applications, testing is not enough, critical contracts should be audited by computer security professionals before being deployed. As the aim of this thesis is to document a proof-of-concept DDoS mitigation application, no audit will be conducted.

\section{Security considerations with Solidity}
Code vulnerabilities are more common in Ethereum than in other environments. The whole blockchain data is public, and serious contracts are made open source in order to create trust for the people who use it. Therefore applications are auditable and bugs are more likely to be found.

The application must also regulate itself and can in most cases not recover from an attack with an update.

In addition to that, the language Solidity aims to have a syntax similar to JavaScript and is therefore quite loose with enforcing the correct type. External static code analysis tools such as Solium \cite{Solium} do not yet have as many rules included as comparable tools for more mature languages.

In this section, some common mistakes that can lead to security exploits will be explained, partially taken from a list compiled by the co-founder of Ethereum \cite{ThinkingAboutSmartContractSecurity} and based on real-word vulnerabilities.

\subsection{Constructor naming exploit}
Simple human mistakes can lead to the contract being vulnerable. The 'Rubixi' contract has a different constructor name than contract name. Therefore the constructor does not get called on deployment, but can be called as a transaction. If done so, the caller of the transaction becomes the owner of the contract. Because of more unfortunate design of the contract, the bug became not immediately apparent and the Solidity contract was correct.

\definecolor{dkgreen}{rgb}{0,0.6,0}
\definecolor{gray}{rgb}{0.5,0.5,0.5}
\definecolor{mauve}{rgb}{0.58,0,0.82}

\lstset{frame=tb,
  language=Java,
  aboveskip=3mm,
  belowskip=3mm,
  showstringspaces=false,
  columns=flexible,
  basicstyle={\small\ttfamily},
  numbers=none,
  numberstyle=\tiny\color{gray},
  keywordstyle=\color{blue},
  commentstyle=\color{dkgreen},
  stringstyle=\color{mauve},
  breaklines=true,
  breakatwhitespace=true,
  tabsize=3
}

\lstinputlisting{snippets/rubixi.sol}


\subsection{Public data}
A misconception is that stored data can be made private on the blockchain. A rock-paper-scissor contract which people used for gambling turned out to have a trivial flaw where the first move could be seen - rock would have the value \texttt{0x60689557} and scissor and paper have a different one.

The main takeaway is that the stored IP addresses will be public (although maybe obfuscated) to everybody. Even attackers can determine the list of IPs that are blocked. The contract should be created in a way that minimises the usefulness for attackers, for example the pattern of the IPs should not make the way how IPs are being blocked predictable.

\subsection{Loops}
The section "Security Considerations" of the Solidity documentation warns about loops that have a non-fixed amount of iterations. With a high number of iterations, the block gas limit could be reached and according to the Solidity documentations "cause the complete contract to be stalled at a certain point".

In the reference contract, the function \texttt{blockIPv4()} has a loop whose amount of iterations are dependant of a transaction argument. The contract has to be tested with a big payload and a solution needs to be developed.

Furthermore, even for constant functions (which are executed locally and have no gas limit), there can be bugs. If, in the reference contract there would be a slight change:

\lstdefinelanguage{diff}{
  morecomment=[f][\color{blue}]{@@},     % group identifier
  morecomment=[f][\color{red}]-,         % deleted lines 
  morecomment=[f][\color{green}]+,       % added lines
  morecomment=[f][\color{magenta}]{---}, % Diff header lines (must appear after +,-)
  morecomment=[f][\color{magenta}]{+++},
}

\lstset{frame=tb,
  language=diff,
  aboveskip=3mm,
  belowskip=3mm,
  showstringspaces=false,
  columns=flexible,
  basicstyle={\small\ttfamily},
  numbers=none,
  numberstyle=\tiny\color{gray},
  keywordstyle=\color{blue},
  commentstyle=\color{dkgreen},
  stringstyle=\color{mauve},
  breaklines=true,
  breakatwhitespace=true,
  tabsize=3
}

\lstinputlisting{snippets/loop-bug.diff}


a bug would be introduced. \texttt{var} would be interpreted as \texttt{uint8}, if there are more than 255 entries in the \texttt{drop\_src\_ipv4} array, an overflow leading to an infinite loop would happen.
This bug can be avoided by disallowing \texttt{var} in the code or by being aware of the issue.

\subsection{Re-Entry bugs using .call()}

The \texttt{.call()} method in Solidity is dangerous as it can execute code from external contracts. This weakness was first exploited in the DAO smart contract, which had over 50 million USD stored in it. The takeaway from the incident is that \texttt{.call()} can call any public function, even the function from which it was called from, leading to recursion.
The following example is not safe:

\definecolor{dkgreen}{rgb}{0,0.6,0}
\definecolor{gray}{rgb}{0.5,0.5,0.5}
\definecolor{mauve}{rgb}{0.58,0,0.82}

\lstset{frame=tb,
  language=Java,
  aboveskip=3mm,
  belowskip=3mm,
  showstringspaces=false,
  columns=flexible,
  basicstyle={\small\ttfamily},
  numbers=none,
  numberstyle=\tiny\color{gray},
  keywordstyle=\color{blue},
  commentstyle=\color{dkgreen},
  stringstyle=\color{mauve},
  breaklines=true,
  breakatwhitespace=true,
  tabsize=3
}

\lstinputlisting{snippets/reentry-bug.sol}


For illustrating the vulnerability, consider that the balance of an account is 10 and that account calls the withdraw method. The account can execute arbitrary code when \texttt{msg.sender.call()} is called on line 3, so \texttt{withdraw()} can also be called again before line 4 of the above snippet is reached. An attacker could withdraw more than what its balance is. The vulnerability could be mitigated by using \texttt{.send()} instead of \texttt{.call()} in the example above.

\subsection{Call stack depth attack}

The call stack depth gets increased when a function calls another function, and if a function returns, the call stack depth gets decreased. A long call stack can be produced by excessive recursion. Solidity has a 1024 call stack depth limit - this means that Solidity could for example not compute the 1025th fibonacci number using recursion.

The deterministic depth limit of Solidity proves to be an attack vector. An attacker could craft a function that calls itself 1023 times and on the 1024th time calls another, vulnerable function, that stops execution as soon as a subfunction is called, because the maximum call stack size is reached. 

An attacker might force a function to only partially execute, which is a problem, if for example a withdraw function is composited of a subfunction that sends funds and a function that decreases the users balance.
The core development team of Ethereum wants to solve this problem on a language-level in EIP \#150 \cite{EIP150} soon, therefore it will not be further discussed here.
