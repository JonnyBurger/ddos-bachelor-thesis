\section{Contract 1: Native storage}
The first variant stores all reports in the blockchain natively. No optimizations regarding speed and cost are being made, all IP addresses and metadata are simply stored in an array.

All the code in this section is assumed to be in the contract body:
\lstinputlisting{snippets/array-store-shell.sol}

Inside the contract body, two structs are defined, with syntax resembling that of the C programming language:

\lstinputlisting{snippets/structs.sol}

For each IP address, a mask can be specified. This makes it possible to specify a range of IP addresses with no redundancy. 
When discussing masks, the notation of '127.0.0.0/24' is used, where everything before the slash represents the IP address base and the number behind the slash represented the mask. Assuming IPv4, a mask of '/32' means specifically and only that IP, while '/0' means the whole range of IP addresses possible. For example, '127.0.0.0/24' means all IP addresses from 127.0.0.\textbf{0} to 127.0.0.\textbf{255} (all addresses that match the first 24 bits if the IP address base). In IPv4, the maximum value for a mask is 32, in IPv6, the maximum value for a mask is 128.

An entry that can be added to the smart contract is a composite of 3 values: The 'victim IP' or destination IP, the 'attacker IP' or source IP and an expiration date. Expired reports can be filtered by comparing to the \texttt{block.timestamp} global variable.

\clearpage

The next part of the contract is the constructor function.

\lstinputlisting{snippets/array-store-constructor.sol}

The constructor function takes two arguments, an IP address and a mask, which form the 'IP Boundary'. The boundary makes it possible for the creator of the smart contract to restrict the destination IP addresses that can be added to only a certain range. This concept is taken from the original paper \cite{OriginalPaper}.

The address of the creator of the contract instance gets stored in the \texttt{owner} property. This makes it possible to write access control logic in other parts of the contract.

The constructor uses a 'modifier' called \texttt{needsMask}. A modifier is a piece of code that is being run before the method body. The modifier \texttt{needsMask} simply throws when the user calls the constructor without the second argument (which the language itself allows).

If the second argument is missing, the mask would default to 0, encapsulating all IP addresses possible. A user of the smart contract could inadvertently give permission to an user to register reports for the whole range of IP addresses by forgetting a method argument, hence the check using the modifier.

The underscore statement in Solidity can only be used in modifiers. Its effect is that it jumps to the main method body immediately.

The following method is for adding a 'customer'.

\lstinputlisting{snippets/array-store-create-customer.sol}

Only the owner of the contract can add a customer, otherwise the method throws an error. In addition to checking ownership of the contract, the contract also checks if the mask argument was supplied using the previously discussed \texttt{needsMask} modifier.

The method also checks if the supplied IP range is outside the IP boundary and throws if this is the case. For that, if the mask of the IP boundary is \texttt{n}, the last 128 - \texttt{n} bits of both IP addresses are set to 0 and then both IP addresses are compared to each other. For example, to find out if \texttt{::127.0.200.20/120} is in the \texttt{::127.0.0.1/112} boundary, the last 16 bits (128 - 112) are set to zero in both addressed. Then, because \texttt{::127.0.0.0 = ::127.0.0.0}, it is true that the first IP range is included in the second one. An additional check has the be made whether the supplied mask has a smaller numerical value than the mask of the IP boundary. If this is the case, then it is automatically a violation because it cannot be a subset of the IP boundary.

The following method provides an interface for registering a report:

\lstinputlisting{snippets/array-store-block-fn.sol}

Two cases are distinguished: If the owner of the smart contract calls the method, the rule gets applied to the whole IP boundary. Otherwise, the rule gets applied to the range of IP addresses that was registered using the \texttt{createCustomer} method.

The creator of the smart contract can restrict for which IP address ranges the customer can add reports, but the customer can add reports in that range without contacting the smart contract owner. The correct permissions are verified by the other blockchain users who are executing the transaction also and updating their state of the blockchain.

This code is enough to allow for customers adding reports to the contract. Because of a technicality, the reports array can not be marked as \texttt{public}, because public nested structs are not supported in Solidity at the time of writing. For programming with Solidity, it is generally advised to keep the data structure as flat as possible to avoid this problem.
Although all data in a contract is technically public, it is complicated to access, as disassembly of native blockchain data is required. In order to create an interface where blockchain users can read nested structs, it is necessary to flatten the structure into one-dimensional arrays.

\lstinputlisting{snippets/array-store-filter.sol}

The \texttt{blocked} function calls helpers functions which are also declared in the contract.

\lstinputlisting{snippets/array-store-helper-fn.sol}

The helper functions \texttt{filter} and \texttt{isNotExpired} composed together form the \texttt{getUnexpired} function which returns all reports that are not yet expired. No lambda functions or arrays of dynamic size are supported in Solidity. Therefore, two for-loops are needed, the first to determine the size of the array that should be created, and the second one to fill an array of that size. This does not make the contract more expensive to operate, since all the methods are marked as \texttt{internal} and are only invoked locally.

The contract now has all methods needed to write and read reports. These methods can be called programmatically using a client library, like \texttt{geth} (the Go client) or \texttt{web3.js} (the Javascript client), or using a graphical interface like Mist \cite{Mist}.

For inserting IPv6 addresses into the contract from a client interface, it needs to be converted into a 128-bit integer. Helper libraries exist for this task, for example the \texttt{ip-address} package on the npm (Node package manager) registry \cite{IpAddressNpm} allows to easily convert a string representation of an IP address to a big integer and vice versa:

\lstinputlisting{snippets/ip-address-npm.js}
