\chapter{Introduction}

\section{Blockchains and Ethereum}

A Blockchain is a decentralized database consisting of a chain of cryptographically secured units, called 'blocks'. Each block references the previous block and cannot be modified without breaking the subsequent blocks. A blockchain is continuously growing, as new data is inserted at the end of the chain.
The most popular application for blockchains are digital crypto currencies, the most widely used implementation is Bitcoin. With Bitcoin, which had it's breakthrough in 2008, network users can exchange tokens securely over a completely decentralized protocol. Because of this usefulness, these tokens are valued with real money, the total market capitalization of Bitcoin is several dozen millions.
 
Ethereum \cite{Ethereum} is blockchain protocol that is inspired by Bitcoin, but not only allows for sending and receiving of tokens, but also offers a scripting language called Solidity, which allows anyone to write programs which can be run on the blockchain. Examples for applications that could run on Ethereum are games like Tic-Tac-Toe or Poker, finance applications like venture capital funds and Initial Coin Offerings (a company raising funds by selling shares of it to investors).
An application is also called a smart contract and allows for storage of arbitrary information and allows for users to send transactions that mutate the storage. The creator of the smart contract controls with code the permissions of the users, the conditions and behaviors of the mutations. This enables a wide variety of possible applications, including a collaborative DDoS mitigation solution, which this thesis is about.

\section{Denial of Service and DDoS}

A Denial of Service (DoS) is when a machine or network resource that should be online is being disrupted \cite{USCERT}. An attacker can either force a DoS by crafting a request payload causing a lot of computational work on the target machine or by flooding the target with requests. The motivation behind a DoS attack is that the attacker sees benefit in the victim's service being disrupted, be that disagreement with the service offered (activist attack), that the service is from a competitor, or that taking down a service brings pleasure to the victim \cite{DDoSOverview}.

A Distributed Denial of Service (DDoS) attack is a DoS attack where the requests are coming from many different sources. By distributing the requests, a denial of service attack can reach much higher magnitudes in terms of traffic and can become much harder to control.
Usually, an attacker takes control of as many internet-connected devices as possible by spreading malware, and then directing these devices to attack the victim.


\section{Motivation}

The amount of DDoS attacks globally is on the rise \cite{DDoSRise} and mitigation is happening only with limited success. 
DDoS protection is a burden for most services and requires a lot of human and financial resources, such that it is hard to justify for many services to invest in DDoS protection. A standard tool for signaling DDoS attacks that can be used collaboratively would lower the investment needed to prepare for DDoS attacks.
The Ethereum blockchain is a database that is already available and that can not be taken down \cite{Ethereum}. With Solidity, the blockchain is scriptable and interfaces for storing and retrieving IP addresses can be programmed. With Ethereum being an readily available infrastructure independent from web services, it opens up an opportunity for storing signals of IP addresses.

Existing DDoS signaling systems such as described in \cite{IETFDraft} send messages about attack information in key-value form using server infrastructure. The Ethereum blockchain allows to signal messages that have a similar format. This presents a chance to decrease the infrastructure

\section{Description of Work}

The paper {"}A Blockchain-based Architecture for Collaborative DDoS Mitigation with Smart Contracts and SDN{"} \cite{OriginalPaper} proposes to use the Ethereum blockchain as a registry for IP addresses from which attacks are originating from. The data can then be read by parties like ISPs who can filter out the malicious packets before they even reach the victim of the attack. This eliminates the need for additional architecture.

Three variants of a smart contract will be developed and compared to each other. Each smart contract serves the same purpose, the storage of a list of IP addresses plus relevant metadata. All variants accept the input of IP addresses and allow to read from it, although the storage of the information differs.

The three variants are:

1. A smart contract that stores a list of all IP addresses on the blockchain, similar to the contract shown in the original paper \cite{OriginalPaper} (Listing 1-3). The difference is that it supports more features and is more robust, closer to a prototype from a proof of concept.

2. A contract that points to a static web resource containing all the information. This will also include the design of the additional infrastructure needed for this variant of the contract.

3. A contract that implements a bloom filter as a mean of reducing cost and space.

All contracts should support both whitelists and blacklists.
A whitelist, in this case, is a list of IP addresses that are explicitly allowed to access the server, while a blacklist is a list of IP addresses that are explicitly disallowed to access the service.
Both IPv4 and IPv6 addresses should be insertable.

It should be made as easy as possible to modify the list. Additionally, the contract should make it possible to easily verify the identity of the reporter and prevent unauthorized modifications of entries on behalf of others.

The smart contracts will be benchmarked for speed and cost. In addition to comparing numbers, opinions about ease of use and general suitability for the job are expressed.

For the conclusions, a winner is picked and a statement is made whether the experiment was positive. A look into the future, including further work needed and the development of the Ethereum ecosystem is given.

\section{Thesis Outline}

\textbf{Chapter 1:} Discusses related work. 

\textbf{Chapter 2} The characteristics of Ethereum smart contracts and their implications on our implementation are discussed, as well as the properties and mechanics of our solution and look at security risks.

\textbf{Chapter 3:} The smart contracts based on the planning in chapter 1 are being implemented. The chapter describes the implementation technique, development process, testing strategy and documents the established protocols.

\textbf{Chapter 4:} A generic cost model for smart contracts is being introduced to enable the estimation of costs for the developed smart contracts. This chapter also discusses the pricing mechanism of Ethereum and covers how transaction speed is relasted to cost.

\textbf{Chapter 5:} The developed smart contracts are benchmarked for cost, speed and volume. 

\textbf{Chapter 6:} A recommendation is made for a smart contract variant and a statement is made on whether a blockchain-based approach to mitigating DDoS attacks is suitable for real-world use.
