
\chapter*{Abstract}
\addcontentsline{toc}{chapter}{Abstract}

\selectlanguage{german}

Attacken wie Distributed Denial-of-Service (DDoS) stellen ein immer gr{"o}sser werdende Gefahr dar f{"u}r Computernetzwerke und Internet-Services.
Existierende Strategien zur Bek{"a}mpfung von DDoS-Attacken sind ineffizient aufgrund mangelnder Ressourcen und Inflexibilit{"a}t.
Blockchains wie Ethereum erm{"o}glichen neue Methoden zur Mitigation von DDoS-Attacken. Mittels Smart Contract k{"o}nnen IP-Addressen von Attackierern auf einer dezentralisierten Plattform signalisiert werden, ohne zus{"a}tzliche Infrastruktur einzusetzen.
Diese Arbeit dokumentiert die Entwicklung mehrerer Smart Contracts zur Signalisierung von DDoS-Attacken und vergleicht sie, bespricht die Ethereum-Umgebung und ihre Auswirkungen auf die Architektur, gibt Auskunft {"u}ber Leistung sowie Kosten und evaluiert die Machbarkeit und Wirksamkeit einer blockchainbasierten L{"o}sung zur Bek{"a}mpfung von DDoS-Attacken.

\selectlanguage{english}
