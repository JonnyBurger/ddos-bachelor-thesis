
\chapter*{Zusammenfassung}
\addcontentsline{toc}{chapter}{Zusammenfassung}

\selectlanguage{german}

Attacken wie Distributed Denial-of-Service (DDoS) stellen eine immer gr{"o}sser werdende Gefahr dar f{"u}r Computernetzwerke und Internet-Services.
Existierende Strategien zur Bek{"a}mpfung von DDoS-Attacken sind ineffizient aufgrund mangelnder Ressourcen und Inflexibilit{"a}t.
Blockchains wie Ethereum erm{"o}glichen neue Methoden zur Mitigation von DDoS-Attacken: Mittels Smart Contracts k{"o}nnen IP-Addressen von Attackierern auf einer dezentralisierten Plattform signalisiert werden, ohne zus{"a}tzliche Infrastruktur einzusetzen.
Diese Arbeit dokumentiert die Entwicklung mehrerer Smart Contracts zur Signalisierung von DDoS-Attacken und vergleicht sie, bespricht die Ethereum-Umgebung und ihre Auswirkungen auf die Architektur, gibt Auskunft {"u}ber Leistung sowie Kosten und evaluiert die Machbarkeit und Wirksamkeit einer blockchainbasierten L{"o}sung zur Bek{"a}mpfung von DDoS-Attacken.

\selectlanguage{english}
\chapter*{Abstract}
\addcontentsline{toc}{chapter}{Abstract}
Attacks like Distributed Denial-of-Service (DDoS) pose a growing threat to computer networks and internet services.
Existing strategies for mitigating DDoS attacks are inefficient because of lacking resources and inflexibility.
Blockchains like Ethereum enable new ways to fight DDoS attacks: Using smart contracts, IP addresses of attackers can be singalized without additional infrastructure.
This work documents the development of multiple smart contracts for signalisation of DDoS attacks and compares them, describes the Ethereum environment and its effect on the smart contract architecture, gives information and advice about the performance and costs and evaluates the overall feasability and effectivity of a blockchain-based solution for fighting DDoS attacks.
